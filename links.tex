\documentclass[11pt]{article}
\usepackage[T1]{fontenc}
\usepackage[utf8]{inputenc}
\usepackage[french]{babel}
\usepackage[colorlinks=true]{hyperref}
\usepackage{mathrsfs}
\usepackage{geometry}
%\usepackage{fullpage}
\usepackage{listings}
\usepackage{xcolor,graphicx}
\usepackage{xspace}
\usepackage{verbatim}
\usepackage{textcomp}
\usepackage{amsmath}
\usepackage{amsfonts}
\usepackage{syntax}
\usepackage{amssymb}

\title{Compiling queries in \links}
\author{Gabriel \textsc{Radanne}}
\date{\today}

\newcommand\mysc[1]{{\rmfamily\textsc{#1}}\xspace}
\newcommand\links{\mysc{Links}}
\newcommand\sql{\mysc{SQL}}
\newcommand\js{\mysc{Javascript}}

\newcommand\sig[1]{\tt\bf #1}
\newcommand\linkslst[1]{\tt #1}

\begin{document}
\maketitle
\newpage

\section{A quick introduction to Links}

\subsection{What is Links ?}
In a nutshell, Links is a web functional programming language. We can define Links by two goals.

\links aim to avoid code separation between the client, the server and the database. In usual web programming, you keep those three component separated : for example, the client is written in \js, the client in \mysc{PHP} and queries are in \sql. 
These separation lead to what's called {\it the independence mismatch} problem : How to be sure than the type of data get by the \sql query fit what the server expect ? 

One of the first \links' related article \cite{links:tiers} was called ``web programming without tiers'' : In \links, there is only one language for the client, the server and the queries. 
The strong type system ensure that data traveling between those three parts will properly work together. 
The language itself will decide if the function should be on the server or the client side, ensuring data availability.\\

Another goal of \links is to use the functional paradigm and a strong type system to ease web programming with usefull abstractions. \cite{links:formlets} and \cite{links:effect} introduce news concepts to manipulate form components and queries in a strongly typed way.

In this report, we will mostly be interested by \links query system.

\subsection{The Query system}

Queries in \links are integrated as an element of the language. As opposed to most server-side language, you don't write sql strings : you just write usual code and it will be translated into an \sql query at runtime.
This have two consequences :
\begin{itemize}
\item The language must ensure than every functions that appear inside a query can actually be translated. For example you can't translate recursive functions to \sql.
\item Each function can have up to two meanings : a {\em query} meaning and a {\em language} meaning.
\end{itemize}

Effects, as described in \ref{links:effect}, allow \links' type system to handle those specificity nicely.



\section{Links's insides}

\subsection{Query normalization}

\subsection{The internal representation}

\subsection{The Irtoml module}


\section{Query compilation in Links}

\subsection{Motivations}

\subsection{Issues raised by query compilation}

\subsection{Doubleling and splicing}

\subsection{Implementation}


\section{Test \& benchmarks}

\subsection{Working exemple}

\subsection{Benchmarks}

\section{Conclusion}

\begin{thebibliography}{99}
\bibitem{links:tiers} \href{http://groups.inf.ed.ac.uk/links/papers/links-fmco06.pdf}{Links: web programming without tiers} Ezra Cooper, Sam Lindley, Philip Wadler \and Jeremy Yallop
\bibitem{links:formlets} \href{http://groups.inf.ed.ac.uk/links/papers/formlets-essence.pdf}{The essence of form abstraction} Ezra Cooper, Sam Lindley, Philip Wadler \and Jeremy Yallop
\bibitem{links:effect} \href{http://homepages.inf.ed.ac.uk/slindley/papers/corelinks.pdf}{Row-based effect types for database integration} Sam Lindley \and James Cheney
\end{thebibliography}


\end{document}
